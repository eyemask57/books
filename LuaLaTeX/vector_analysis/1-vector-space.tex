\chapter{ベクトル空間}

\section{基礎}

\begin{dfn}{ベクトル空間の公理}{vector-space}
集合$V$,体$F$,$\bm{x},\bm{y},\bm{z} \in V$,$a,b \in F$とする.

加法${+}:V \times V \to V$,スカラー倍${*} : F \times V \to V $について次の性質を満たすとき,
組$(V,{+},{*})$は$F$上の
\index{vector space}\index{べくとるくうかん@ベクトル空間}\emph{ベクトル空間(vector space)}
または
\index{linear space}\index{せんけいくうかん@線形空間}\emph{線形空間(linear space)}
という.
$V$の元を
\index{vector}\index{べくとる@ベクトル}\emph{ベクトル(vector)}といい,
$F$の元を
\index{scalar}\index{すからー@スカラー}\emph{スカラー(scalar)}という.
\begin{enumerate}
	\item $\bm{x} + \bm{y} = \bm{y} + \bm{x}$
	\item $(\bm{x} + \bm{y}) + \bm{z} = \bm{x} + (\bm{y} + \bm{z})$
	\item ある元$\bm{0} \in V$が存在し,$\bm{0} + \bm{x} = \bm{x}$となる.
	\item 任意の$\bm{x}$に対して,ある元$\bm{x}'$が存在し,$\bm{x} + \bm{x}' = \bm{x}' + \bm{x} = \bm{0}$となる.$\bm{x}'$のことを$\bm{x}$の\index{inverse element}\index{ぎゃくげん@逆元}\emph{逆元(inverse element)}という.
	\item $a(\bm{x} + \bm{y}) = a\bm{x} + a\bm{y}$
	\item $(a + b) \bm{x} = a\bm{x} + b\bm{y}$
	\item $a(b\bm{x})=(ab)\bm{x}$
	\item $F$の単位元$1 \in F$について,$1\bm{x} = \bm{x}$となる.
\end{enumerate}
\end{dfn}

今回は$V = \mathbb{R}^n , F = \mathbb{R}$,特に$n = 2,3$の場合を扱う.
また,表記を省略して,単に$\mathbb{R}^n$がベクトル空間であると書くことがある.

ベクトル$\bm{x} \in \mathbb{R}^n$について,
各成分を明示して記述する方法をベクトルの$\bm{x}$の\emph{成分表示(component representation)}という.
$x_i \in \mathbb{R}\ (i = 1,\ldots,n)$を用いて,
\[
\bm{x} = (x_1,\ldots,x_n)=
\begin{pmatrix}
	x_1\\
	\vdots\\
	x_n
\end{pmatrix}
\]
と書く.$n \times 1$行列として同一視することもある.

$\bm{x},\bm{y} \in \mathbb{R}^n$とその成分表示$x_i,y_i \in \mathbb{R}\ (i = 1,\ldots,n)$
および$a \in \mathbb{R}$について,次のように演算を定義すれば$\mathbb{R}^n$は$\mathbb{R}$上のベクトル空間となる.

\[
\bm{x}+\bm{y}\coloneqq\begin{pmatrix}
	x_1 + y_1\\
	\vdots\\
	x_n + y_n
\end{pmatrix}\quad,\qquad
a\bm{x}\coloneqq
\begin{pmatrix}
	ax_1\\
	\vdots\\
	ax_n
\end{pmatrix}
\]

\begin{dfn}{ゼロベクトル}{zero-vector}
	全ての成分が$0$であるベクトルのことを
	\index{zero vector}\index{ぜろべくとる@ゼロベクトル}\emph{ゼロベクトル(zero vector)}といい,
	$\bm{0}\coloneqq(0,\ldots,0) \in \mathbb{R}^n$と表す.
\end{dfn}

\begin{dfn}{数ベクトルの相等}{equiv-vector}
	$\bm{a},\bm{b}\in\mathbb{R}^n$とする.
	2つのベクトル$\bm{a}=(a_1,\ldots,a_n),\bm{b}=(b_1,\ldots,b_n)$が等しいとは,
	\[
	a_i = b_i \quad (i = 1,\ldots,n)
	\]
	が成り立つことをいう.
\end{dfn}

\begin{dfn}{線形結合}{linear-combination}
	$\bm{v}_1,\ldots,\bm{v}_r \in \mathbb{R}^n, k_1,\ldots,k_r \in \mathbb{R}\ (r \in \mathbb{N})$
	について,
	\[
	\bm{v} = k_1\bm{v}_1 + \cdots + k_r\bm{v}_r
	\]
	となるベクトル$\bm{v} \in \mathbb{R}^n$のことを,$\bm{v}_1,\ldots,\bm{v}_r$の
	\index{linear combination}\index{せんけいけつごう@線形結合}
	\emph{線形結合(linear combination)}
	または
	\index{いちじけつごう@一次結合}\emph{一次結合}
	という.
\end{dfn}

\begin{dfn}{線形独立・線形従属}{linearly-independent}
	$\bm{v}_1,\ldots,\bm{v}_r \in \mathbb{R}^n$とする.
	任意の$k_1,\ldots,k_r \in \mathbb{R}\ (r \in \mathbb{N})$について,
	\[
	k_1\bm{v}_1 + \cdots + k_r\bm{v}_r = \bm{0}
	\implies
	k_1 = \cdots = k_r = 0
	\]
	がなりたつとき,$\bm{v}_1,\ldots,\bm{v}_r$は
	\index{linearly independent}\index{せんけいどくりつ@線形独立}
	\emph{線形独立(linearly independent)}または\index{いちじどくりつ@一次独立}\emph{一次独立}であるという.

	$\bm{v}_1,\ldots,\bm{v}_r$が一次独立でないとき,
	$\bm{v}_1,\ldots,\bm{v}_r$は\index{linearly dependent}\index{せんけいじゅうぞく@線形従属}
	\emph{線形従属(linearly dependent)}
	または\index{いちじじゅうぞく@一次従属}\emph{一次従属}という.
\end{dfn}

\section{演算}

\begin{dfn}{標準内積}{inner-product}
	$\bm{a},\bm{b}\in\mathbb{R}^n,\bm{a}=(a_1,\ldots,a_n),\bm{b}=(b_1,\ldots,b_n)$とする.
	\[
	\braket{\bm{a},\bm{b}} \coloneqq \bm{a}^{\mathsf{T}}\bm{b} = a_1 b_1 + \cdots + a_n b_n= \sum_{i=1}^{n}a_i b_i
	\]
	となる演算$\braket{{-},{-}}:\mathbb{R}^n\times\mathbb{R}^n\to\mathbb{R}$を
	\index{canonical inner product}\index{ひょうじゅんないせき@標準内積}
	\emph{標準内積(canonical inner product)}または単に\emph{内積}という.
	ただし,$\bm{a}^{\mathsf{T}}$は$\bm{a}$を$n\times{}1$行列とみなしたときの転置行列で,$1\times{}n$行列である.
	$\bm{a},\bm{b}$の標準内積は$\bm{a} \cdot \bm{b}$と書くこともある.
\end{dfn}

\begin{prop}{標準内積の性質}{inner-product}
	$\bm{x},\bm{y},\bm{z}\in\mathbb{R}^n,c\in\mathbb{R}$とする.
	このとき,次の性質が成り立つ.
	\begin{enumerate}
		\item $\braket{\bm{x},\bm{y}} = \braket{\bm{y},\bm{x}}$\quad(対称性)
		\item $\braket{\bm{x}+\bm{y},\bm{z}} = \braket{\bm{x},\bm{z}} + \braket{\bm{y},\bm{z}}$\quad(線形性)
		\item $\braket{c\bm{x},\bm{y}} = c\braket{\bm{x},\bm{y}}$\quad(線形性)
		\item $\braket{\bm{x},\bm{x}} \geq 0$\quad(正値性)
		\item $\braket{\bm{x},\bm{x}} = 0 \iff \bm{x} = \bm{0}$
	\end{enumerate}
\end{prop}

\begin{proof}
	\begin{enumerate}
		\item
			$
				\braket{\bm{x},\bm{y}}
				= x_1 y_1 + \cdots + x_n y_n
				= y_1 x_1 + \cdots + y_n x_n
				= \braket{\bm{y},\bm{x}}
			$
		\item
			$\braket{\bm{x}+\bm{y},\bm{z}}
			= (x_1 + y_1) z_1 + \cdots + (x_n + y_n) z_n$\\
			$= x_1 z_1 + \cdots + x_n z_n + y_1 z_1 + \cdots + y_n z_n
				= \braket{\bm{x},\bm{z}} + \braket{\bm{y},\bm{z}}$
		\item
			$
				\braket{c\bm{x},\bm{y}}
				= (cx_1) y_1 + \cdots + (cx_n) y_n
				= c(x_1 y_1 + \cdots + x_n y_n)
				= c\braket{\bm{y},\bm{x}}
			$
		\item
			${x_i}^2 \geq 0 \ (i = 1,\ldots,n)$より$\displaystyle\sum_{i=1}^{n}{x_i}^2 \geq 0$
		\item
			$\displaystyle\sum_{i=1}^{n}{x_i}^2 = 0,{x_i}^2 \geq 0\ (i = 1,\ldots,n)$
			より${x_i}^2 = 0$,よって$x_i = 0$
	\end{enumerate}
\end{proof}

\begin{dfn}{直交・平行}{orthogonal-parallel}
	$\bm{x},\bm{y} \in \mathbb{R}^n$とする.

	$\braket{\bm{x},\bm{y}}=0$であるとき,$\bm{x}$と$\bm{y}$は
	\index{orthogonal}\index{ちょっこう@直交}
	\emph{直交(orthogonal)}しているといい,
	$\bm{x}\perp\bm{y}$と書く.

	$\bm{x} = k\bm{y}\ (k \neq 0)$となるスカラー$k\in\mathbb{R}$が存在するとき,
	$\bm{x}$と$\bm{y}$は
	\index{parallel}\index{へいこう@平行}
	\emph{平行(parallel)}であるといい,
	$\bm{x}/\!/\bm{y}$と書く.
\end{dfn}

\begin{dfn}{Euclidノルム}{Euclidean-norm}
	$\bm{a}\in\mathbb{R}^n,\bm{a}=(a_1,\ldots,a_n)$とする.
	\[
	\|{\bm{a}}\| \coloneqq \sqrt{\braket{\bm{a},\bm{a}}}
	= \sqrt{{a_1}^2 + \cdots + {a_n}^2}
	= \sqrt{\sum_{i=1}^{n}{a_i}^2}
	\]
	となる演算$\|{-}\|:\mathbb{R}^n\to\mathbb{R}$を
	\index{Euclidean norm}\index{ゆーくりっどのるむ@Euclidノルム}
	\emph{Euclidノルム(Euclidean norm)}または単に\emph{ノルム}という.
	$\|{\bm{a}}\|$を$\bm{a}$の\emph{大きさ}という.
\end{dfn}

\begin{prop}{Euclidノルムの性質}{Euclidean-norm}
	$\bm{x},\bm{y},\in \mathbb{R}^n,c \in \mathbb{R}$とする.
	このとき,次の性質が成り立つ.
	\begin{enumerate}
		\item $\|\bm{x}\| \geq 0$
		\item $\|\bm{x}\| = 0 \iff \bm{x} = 0$
		\item $\|c\bm{x}\| = |c| \|\bm{x}\|$
	\end{enumerate}
\end{prop}

\begin{proof}
	(1),(2)は\propref{inner-product}(4),(5)より自明なので省略.
	\begin{enumerate}[start=3]
		\item $\displaystyle\|c\bm{x}\|
		= \sqrt{(cx_1)^2 + \cdots + (cx_n)^2}
		= \sqrt{c^2({x_1}^2 + \cdots + {x_n}^2)}
		= |c| \|\bm{x}\|
		$
	\end{enumerate}
\end{proof}

\begin{dfn}{単位ベクトル}{unit-vector}
	$\|\bm{e}\|=1$を満たすベクトル$\bm{e}\in\mathbb{R}^n$のことを
	\index{unit vector}\index{たんいべくとる@単位ベクトル}
	\emph{単位ベクトル(unit vector)}という.

	また,第$i$成分のみが$1$であり,その他の成分が$0$であるような
	ベクトル$\bm{e}_i \in \mathbb{R}^n\ (i=1,\ldots,n)$のことを
	\index{unit basic vector}\index{きほんたんいべくとる@基本単位ベクトル}
	\emph{基本単位ベクトル(unit basic vector)}という.
\end{dfn}

$\mathbb{R}^2$のとき,$x$軸,$y$軸の方向を向いた基本単位ベクトルをそれぞれ次のように書く.
\[
\bm{e}_x =
\bm{e}_1 =
\begin{pmatrix}
	1\\0
\end{pmatrix}
\quad,\quad
\bm{e}_y =
\bm{e}_2 =
\begin{pmatrix}
	0\\1
\end{pmatrix}
\]

$\mathbb{R}^3$のとき,$x$軸,$y$軸,$z$軸の方向を向いた基本単位ベクトルをそれぞれ次のように書く.
\[
	\bm{e}_x =
	\bm{e}_1 =
	\begin{pmatrix}
		1\\0\\0
	\end{pmatrix}
	\quad,\quad
	\bm{e}_y =
	\bm{e}_2 =
	\begin{pmatrix}
		0\\1\\0
	\end{pmatrix}
	\quad,\quad
	\bm{e}_z =
	\bm{e}_3 =
	\begin{pmatrix}
	0\\0\\1
	\end{pmatrix}
\]

\begin{dfn}{ベクトルのなす角}{vector-angle}
	$\bm{a},\bm{b} \in \mathbb{R}^n$とする.
	\[\theta = \arccos \frac{\braket{\bm{a},\bm{b}}}{\|\bm{a}\|\|\bm{b}\|}\]
	を満たす$\theta \in \mathbb{R}\ (0\leq \theta \pi)$のことを$\bm{a}$と$\bm{b}$の
	\index{angle}\index{なすかく@なす角}
	\emph{なす角(angle)}という.
	また,
	\begin{align*}
		&\cos \theta = \frac{\braket{\bm{a},\bm{b}}}{\|\bm{a}\|\|\bm{b}\|}\\
		&\sin \theta = \sqrt{1 - (\cos \theta)^2}
		= \sqrt{1 - \left(\frac{\braket{\bm{a},\bm{b}}}{\|\bm{a}\|\|\bm{b}\|}\right)^2}
	\end{align*}である.
\end{dfn}

\begin{prop}{Cauchy--Schwarzの不等式}{Cauchy--Schwarz-inequality}
	$\bm{x},\bm{y},\in \mathbb{R}^n$とする.このとき,
	\[|\braket{\bm{x},\bm{y}}| \leq \|\bm{x}\|\|\bm{y}\|\]
	が成り立つ.この不等式のことを
	\index{Cauchy--Schwarz inequality}\index{こーしーしゅわるつのふとうしき@Cauchy--Schwarzの不等式}
	\emph{Cauchy--Schwarzの不等式(Cauchy--Schwarz inequality)}という.
\end{prop}

\begin{proof}
	$t \in \mathbb{R}$とする.\propref{inner-product}(4),(5)より
	\[
		0 \leq \braket{\bm{x} + t\bm{y},\bm{x} + t\bm{y}}
		= \braket{\bm{x},\bm{x}} + 2t\braket{\bm{x},\bm{y}} + t^2\braket{\bm{y},\bm{y}}
	\]

	上記の二次方程式は重解または虚数解をもつ.よって,2次方程式の判別式$D \leq 0$より,
	\[
		0 \geq \frac{D}{4} = \braket{\bm{x},\bm{y}}^2 - \braket{\bm{x},\bm{x}}\braket{\bm{y},\bm{y}}
	\]

	よって$\braket{\bm{x},\bm{y}}^2 \leq \braket{\bm{x},\bm{x}}\braket{\bm{y},\bm{y}}$.

	$\braket{\bm{x},\bm{y}}^2 = |\braket{\bm{x},\bm{y}}|^2$
	より\[\displaystyle\sqrt{\braket{\bm{x},\bm{y}}^2}\leq \sqrt{\braket{\bm{x},\bm{x}}}\sqrt{\braket{\bm{y},\bm{y}}}\iff|\braket{\bm{x},\bm{y}}| \leq \|\bm{x}\|\|\bm{y}\|\]
\end{proof}

\begin{prop}{三角不等式}{triangle-inequality}
	$\bm{x},\bm{y},\in \mathbb{R}^n$とする.このとき,
	\[\|\bm{x} + \bm{y}\| \leq \|\bm{x}\|+\|\bm{y}\|\]
	が成り立つ.この不等式のことを
	\index{triangle inequality}\index{さんかくふとうしき@三角不等式}
	\emph{三角不等式(triangle inequality)}という.
\end{prop}

\begin{proof}
	両辺を2乗して,左辺から右辺を引く.
	\begin{align*}
		&(\|\bm{x}\| + \|\bm{y}\|)^2 - \|\bm{x} + \bm{y}\|^2\\
		&= \|\bm{x}\|^2 + 2\|\bm{x}\|\|\bm{y}\| + \|\bm{y}\|^2 - \braket{\bm{x}+\bm{y},\bm{x}+\bm{y}}\\
		&= \braket{\bm{x},\bm{x}} + 2\|\bm{x}\|\|\bm{y}\| + \braket{\bm{y},\bm{y}}
			- 2\braket{\bm{x},\bm{y}} - \braket{\bm{x},\bm{x}} - \braket{\bm{y},\bm{y}}\\
		&= 2\|\bm{x}\|\|\bm{y}\| - 2\braket{\bm{x},\bm{y}}
	\end{align*}

	Cauchy--Schwarzの不等式より,$2(\|\bm{x}\|\|\bm{y}\| - \braket{\bm{x},\bm{y}}) \geq 0$.

	よって,$(\|\bm{x}\| + \|\bm{y}\|)^2 \geq \|\bm{x} + \bm{y}\|^2
		\iff \|\bm{x}\| + \|\bm{y}\| \geq \|\bm{x} + \bm{y}\|$.
\end{proof}

\begin{dfn}{Euclid距離}{Euclidean-distance}
	$\bm{a},\bm{b}\in\mathbb{R}^n,\bm{a}=(a_1,\ldots,a_n),\bm{b}=(b_1,\ldots,b_n)$とする.
	\[
	d(\bm{a},\bm{b}) \coloneqq \| \bm{a} - \bm{b} \|
	= \sqrt{{a_1 - b_1}^2 + \cdots + {a_n - b_n}^2}
	= \sqrt{\sum_{i=1}^{n}{(a_i - b_i)}^2}
	\]
	となる演算$d : \mathbb{R}^n\times\mathbb{R}^n\to\mathbb{R}$を
	\index{Euclidean distance}\index{ゆーくりっどきょり@Euclid距離}
	\emph{Euclid距離(Euclidian distance)}または単に\emph{距離}という.
\end{dfn}

\begin{prop}{Euclid距離の性質}{Euclidean-distance}
	$\bm{x},\bm{y}, \bm{z} \in \mathbb{R}^n$とする.
	このとき,次の性質が成り立つ.
	\begin{enumerate}
		\item $d(\bm{x},\bm{y}) \geq 0$\quad (正値性)
		\item $d(\bm{x},\bm{y}) = 0 \iff \bm{x} = \bm{y}$
		\item $d(\bm{x},\bm{y}) = d(\bm{y},\bm{x})$\quad (対称性)
		\item $d(\bm{x},\bm{z})\leq d(\bm{x},\bm{y}) + d(\bm{y},\bm{z})$\quad(三角不等式)
	\end{enumerate}
\end{prop}

\begin{proof}
	(1),(2)は定義より自明なので省略.
	\begin{enumerate}[start=3]
		\item
			\[d(\bm{x},\bm{y}) = \|\bm{x} - \bm{y}\| = |-1|\|\bm{y} - \bm{x}\| = \|\bm{y} - \bm{x}\| = d(\bm{y},\bm{x})\]
		\item
			\[d(\bm{x},\bm{z}) = \|\bm{x} - \bm{z}\| = \| \bm{x} - \bm{y} + \bm{y} - \bm{z} \|\]
			\propref{triangle-inequality}(三角不等式)より,
			\[\| \bm{x} - \bm{y} + \bm{y} - \bm{z}\| \le \| \bm{x} - \bm{y}\| + \|\bm{y} - \bm{z}\|\]
			よって,距離の定義より,
			\[d(\bm{x},\bm{z})\leq d(\bm{x},\bm{y}) + d(\bm{y},\bm{z})\]
	\end{enumerate}
\end{proof}

\begin{dfn}{クロス積}{cross-product}
	$\bm{a},\bm{b} \in \mathbb{R}^3,\bm{a} = (a_1,a_2,a_3),\bm{b}=(b_1,b_2,b_3)$とする.
	\[
		\bm{a} \times \bm{b} \coloneqq
		\begin{pmatrix}
			a_2 b_3 - a_3 b_2\\
			a_3 b_1 - a_1 b_3\\
			a_1 b_2 - a_2 b_1
		\end{pmatrix}
	\]
	となる演算$\times:\mathbb{R}^3\times\mathbb{R}^3\to\mathbb{R}^3$を
	\index{cross product}\index{くろすせき@クロス積}\emph{クロス積(cross product)}または
	\index{vector product}\index{べくとるせき@ベクトル積}\emph{ベクトル積(vector product)},またはしばしば
	\index{outer product}\index{がいせき@外積}\emph{外積(outer product)}という.
\end{dfn}

\begin{prop}{クロス積の性質}{cross-product}
	$\bm{a},\bm{b},\bm{c}\in \mathbb{R}^3,k\in\mathbb{R},\bm{a}=(a_1,a_2,a_3),\bm{b}=(b_1,b_2,b_3),\bm{c}=(c_1,c_2,c_3)$,$\bm{a}$と$\bm{b}$のなす角を$\theta\ (0 \leq \theta \leq \pi)$とする.
	このとき,次の性質が成り立つ.
	\begin{enumerate}
		\item $\braket{\bm{a},\bm{b}\times\bm{c}} = \braket{\bm{b},\bm{c}\times\bm{a}} = \braket{\bm{c},\bm{a}\times\bm{b}} =
			\begin{vmatrix}
				a_1 & a_2 & a_3\\
				b_1 & b_2 & b_3\\
				c_1 & c_2 & c_3
			\end{vmatrix}$
		\item $\bm{a} \times \bm{b} = - \bm{b} \times \bm{a}$
		\item $(k\bm{a}) \times \bm{b} = \bm{a} \times (k\bm{b}) = k(\bm{a} \times \bm{b})$
		\item $\bm{a} \times \bm{a} = \bm{0}$
		\item $\bm{a} \times \bm{b} \perp \bm{a},\ \bm{a} \times \bm{b} \perp \bm{b}$
		\item $\bm{e}_1\times\bm{e}_2 = \bm{e}_3,\ \bm{e}_2\times\bm{e}_3 = \bm{e}_1,\ \bm{e}_3\times\bm{e}_1 = \bm{e}_2$
		\item $\bm{a} \times (\bm{b} + \bm{c}) = \bm{a}\times\bm{b} + \bm{a}\times\bm{c}$
		\item $(\bm{a} + \bm{b}) \times \bm{c} = \bm{a}\times\bm{c} + \bm{b}\times\bm{c}$
		\item $\bm{a} \times \bm{b} = \bm{0} \iff \bm{a} /\!/ \bm{b}$
		\item $(\bm{a} \times \bm{b}) \times \bm{c}
			= \braket{\bm{a},\bm{c}}\bm{b} - \braket{\bm{b},\bm{c}}\bm{a}$
		\item $(\bm{a} \times \bm{b}) \times \bm{c} + (\bm{b} \times \bm{c}) \times \bm{a} + (\bm{c} \times \bm{a}) \times \bm{b} = 0$
		\item $\|\bm{a}\times\bm{b}\|^2 = \|\bm{a}\|^2 \|\bm{b}\|^2 - \braket{\bm{a},\bm{b}}^2$
		\item $\|\bm{a} \times \bm{b}\| = \|\bm{a}\| \|\bm{b}\| \sin \theta$
	\end{enumerate}
\end{prop}
\begin{proof}
	\begin{enumerate}
		\item
			\begin{align*}
				\begin{vmatrix}
					a_1 & a_2 & a_3\\
					b_1 & b_2 & b_3\\
					c_1 & c_2 & c_3
				\end{vmatrix}
				&= a_1 \begin{vmatrix} b_2 & b_3 \\ c_2 & c_3 \end{vmatrix} \braket{\bm{e}_1,\bm{e}_1}
				- a_2 \begin{vmatrix} b_1 & b_3 \\ c_1 & c_3 \end{vmatrix} \braket{\bm{e}_2,\bm{e}_2}
				+ a_3 \begin{vmatrix} b_1 & b_2 \\ c_1 & c_2 \end{vmatrix} \braket{\bm{e}_3,\bm{e}_3}\\
				&=\braket{\bm{a},\bm{b}\times\bm{c}}
			\end{align*}
	\end{enumerate}

	(2)--(9)は省略.

	\begin{enumerate}[start=10]
		\item
			\begin{align*}
				(\bm{a} \times \bm{b}) \times \bm{c}
				&=
				\begin{pmatrix}
					(a_3 b_1 - a_1 b_3)c_3 - (a_1 b_2 - a_2 b_1)c_2\\
					(a_1 b_2 - a_2 b_1)c_1 - (a_2 b_3 - a_3 b_2)c_3\\
					(a_2 b_3 - a_3 b_2)c_2 - (a_3 b_1 - a_1 b_3)c_1
				\end{pmatrix}\\
				&=
				\begin{pmatrix}
					(a_2 c_2 + a_3 c_3)b_1 - (b_2 c_2 + b_3 c_3)a_1\\
					(a_1 c_1 + a_3 c_3)b_2 - (b_1 c_1 + b_3 c_3)a_2\\
					(a_1 c_1 + a_2 c_2)b_3 - (b_1 c_1 + b_2 c_2)a_3
				\end{pmatrix}\\
				&=
				\begin{pmatrix}
					(a_1 c_1 + a_2 c_2 + a_3 c_3)b_1 - (b_1 c_1 + b_2 c_2 + b_3 c_3)a_1\\
					(a_1 c_1 + a_2 c_2 + a_3 c_3)b_2 - (b_1 c_1 + b_2 c_2 + b_3 c_3)a_2\\
					(a_1 c_1 + a_2 c_2 + a_3 c_3)b_3 - (b_1 c_1 + b_2 c_2 + b_3 c_3)a_3
				\end{pmatrix}\\
				&=
				\braket{\bm{a},\bm{c}}\bm{b} - \braket{\bm{b},\bm{c}}\bm{a}
			\end{align*}
		\item
			\begin{align*}
				&(\bm{a} \times \bm{b}) \times \bm{c} + (\bm{b} \times \bm{c}) \times \bm{a} + (\bm{c} \times \bm{a}) \times \bm{b}\\
				&= \braket{\bm{a},\bm{c}}\bm{b} - \braket{\bm{b},\bm{c}}\bm{a}
				+ \braket{\bm{b},\bm{a}}\bm{c} - \braket{\bm{c},\bm{a}}\bm{b}
				+ \braket{\bm{c},\bm{b}}\bm{a} - \braket{\bm{a},\bm{b}}\bm{c}\\
				&= 0
			\end{align*}
		\item
	\end{enumerate}
\end{proof}

\begin{dfn}{Kroneckerのデルタ}{Kronecker-delta}
	$i,j\in\mathbb{N}$とする.

	このとき,以下のように定義される記号を
	\index{Kronecker delta}\index{くろねっかーのでるた@Kroneckerのデルタ}
	\emph{Kroneckerのデルタ(Kronecker delta)}という.
	\[
	\delta_{ij} \coloneqq
	\begin{cases}
		1 & (i = j)\\
		0 & (i \neq j)
	\end{cases}
	\]
\end{dfn}

\begin{dfn}{Levi--Civitaの記号}{Levi--Civita-symbol}
	$i,j,k$はそれぞれ$1,2,3$のいずれかの値を取るとする.

	このとき,以下のように定義される記号を
	\index{Levi--Civita symbol}\index{れヴぃちびたのきごう@Levi--Civitaの記号}
	\emph{Levi--Civitaの記号(Levi--Civita symbol)}という.
	\[
	\varepsilon_{ijk} \coloneqq
	\begin{cases}
		1  & ((i,j,k) = (1,2,3),(2,3,1),(3,1,2))\\
		-1 & ((i,j,k) = (2,1,3),(1,3,2),(3,2,1)) \\
		0  & \text{(otherwise)}
	\end{cases}
	\]
\end{dfn}

\begin{prop}{Levi-Civitaの記号とKroneckerのデルタの公式}{L-K-property}
	$i,j,k,l,m,n$はそれぞれ$1,2,3$のいずれかの値を取るとする.

	このとき,次のことが成り立つ.
	\begin{enumerate}
		\item
			$\braket{\bm{e}_i,\bm{e}_j}=\delta_{ij}$
		\item
			$\braket{\bm{e}_i\times\bm{e}_j,\bm{e}_k}=\varepsilon_{ijk}$
		\item
			$\varepsilon_{ijk}\varepsilon_{lmn} =
				\begin{vmatrix}
					\delta_{il} & \delta_{im} & \delta_{in}\\
					\delta_{jl} & \delta_{jm} & \delta_{jn}\\
					\delta_{kl} & \delta_{km} & \delta_{kn}
				\end{vmatrix}$
		\item
			$\displaystyle\sum_{k=1}^{3}\varepsilon_{ijk}\varepsilon_{lmk}
			= \delta_{il}\delta_{jm}-\delta_{im}\delta_{jl}$
		\item
			$\displaystyle\sum_{j=1}^{3}\sum_{k=1}^{3}\varepsilon_{ijk}\varepsilon_{ljk}
			= 2\delta_{ij}$
	\end{enumerate}
\end{prop}