\hypersetup{
	luatex,
	pdfencoding=auto,
	unicode,
	hidelinks,
	pdfusetitle,
	bookmarks=true,
	bookmarksdepth=3,
	bookmarksnumbered=true,
	colorlinks=true,
	linkcolor=RoyalBlue,
	citecolor=blue,
	urlcolor=gray,
	pdfauthor={eyemask57}
}

\sisetup{separate-uncertainty,per-mode=symbol,detect-all,range-phrase=--}

\ExecuteBibliographyOptions{
	abbreviate=true,
	maxnames=8,
	minnames=3,
	hyperref=auto,
	backref=true,
	abbreviate=true,
	date=year,
	arxiv=abs,
	isbn=true,
	url=true,
	doi=true
}

\setcounter{tocdepth}{2}
\setcounter{section}{-1}

%ソースコード記入
\newcommand{\SourceCode}[2][Source Code]{
\begin{tcolorbox}[title=#1]
\begin{verbatim}
#2
\end{verbatim}
\end{tcolorbox}
}

% 図表見出し
\renewcommand{\tablename}{\textcolor{gray}{▼} 表}
\renewcommand{\figurename}{\textcolor{gray}{▲} 図}

\begin{comment}
\setmainfont{Times-New-Roman}[BoldFont=Times-New-Roman]
\setsansfont{Arial}

\setmainjfont[
	YokoFeatures       = {JFM=jlreq},
	TateFeatures       = {JFM=jlreqv},
	BoldFont           = Yu-Gothic,
	BoldFeatures       = {FakeBold=2},
	ItalicFont         = Yu-Mincho,
	ItalicFeatures     = {FakeSlant=0.33},
	BoldItalicFont     = Yu-Gothic,
	BoldItalicFeatures = {FakeBold=2, FakeSlant=0.33}
	]{Yu-Mincho}
\setsansjfont[
	YokoFeatures       = {JFM=jlreq},
	TateFeatures       = {JFM=jlreqv},
	BoldFont           = Yu-Gothic,
	BoldFeatures       = {FakeBold=2},
	ItalicFont         = Yu-Gothic,
	ItalicFeatures     = {FakeSlant=0.33},
	BoldItalicFont     = Yu-Gothic,
	BoldItalicFeatures = {FakeBold=2, FakeSlant=0.33}
	]{Yu-Gothic}
\end{comment}

\DeclareEmphSequence{\gtfamily\sffamily}

\def\qed{\rightline\square}
\def\qedhere{\quad\square}

\DeclareMathOperator{\sgn}{sgn}
\DeclareMathOperator{\card}{card}
\DeclareMathOperator{\ord}{ord}
\DeclareMathOperator{\Aut}{Aut}
\DeclareMathOperator{\id}{id}
\DeclareMathOperator{\diag}{diag}
\DeclareMathOperator{\spanL}{span}
\DeclareMathOperator{\kernel}{Ker}
\DeclareMathOperator{\image}{Im}
\DeclareMathOperator{\lcm}{lcm}
\DeclareMathOperator{\rank}{rank}
\DeclareMathOperator{\coker}{coker}
\DeclareMathOperator{\Coker}{Coker}
\DeclareMathOperator{\sech}{sech}
\DeclareMathOperator{\csch}{csch}
\DeclareMathOperator{\arcsec}{arcsec}
\DeclareMathOperator{\arccot}{arccot}
\DeclareMathOperator{\arccsc}{arccsc}
\DeclareMathOperator{\arccosh}{arccosh}
\DeclareMathOperator{\arcsinh}{arcsinh}
\DeclareMathOperator{\arctanh}{arctanh}
\DeclareMathOperator{\arcsech}{arcsech}
\DeclareMathOperator{\arccsch}{arccsch}
\DeclareMathOperator{\arccoth}{arccoth}
\DeclareMathOperator{\gradient}{grad}
\DeclareMathOperator{\divergence}{div}
\DeclareMathOperator{\rotation}{rot}
\DeclareMathOperator{\Con}{Con}
\DeclareMathOperator{\Cov}{Cov}
\DeclareMathOperator{\Var}{Var}

\DeclareRobustCommand{\notmid}{\mathrel{\ooalign{$\mkern-5mu\not$\crcr$|$}}}

\definecolor{burgundy}{rgb}{0.5, 0.0, 0.13}
\tcbset{mytheo/.style={fonttitle=\gtfamily\sffamily\bfseries\upshape,
		enhanced,
%		breakable,
		colframe=burgundy,
		colback=burgundy!2!white,
		colbacktitle=burgundy,
		boxrule=0pt,
		borderline south={2pt}{-2pt}{burgundy},
		left*=1\zw,
		right*=1\zw,
		theorem style=standard,
		sharp corners,
		before skip=8pt,
		after skip=10pt,
		before upper={\setlength{\parindent}{1\zw}},
		before lower={\setlength{\parindent}{1\zw}}
}}

\newtcbtheorem[number within=subsection]{thm}{Theorem}%定理
{mytheo}{tm}
\newcommand{\thmref}[1]{{\bfseries\sffamily Theorem~\ref{tm:#1}}}

\newtcbtheorem[use counter from=thm]{prop}{Proposition}%命題
{mytheo}{pro}
\newcommand{\propref}[1]{{\bfseries\sffamily Proposition~\ref{pro:#1}}}

\newtcbtheorem[use counter from=thm]{cor}{Corollary}%系
{mytheo}{co}
\newcommand{\corref}[1]{{\bfseries\sffamily Corollary~\ref{co:#1}}}

%----------------------------------------------------------------------

\newtcbtheorem[use counter from=thm]{dfn}{Definition}%定義
{mytheo,
	colframe=blue!50!black,colback=blue!50!black!2!white,colbacktitle=blue!50!black,borderline south={2pt}{-2pt}{blue!50!black},}{de}
\newcommand{\dfnref}[1]{{\bfseries\sffamily Definition~\ref{de:#1}}}

\newtcbtheorem[use counter from=thm]{ax}{Axiom}%公理
{mytheo,
	colframe=teal!50!black,colback=teal!50!black!2!white,colbacktitle=teal!50!black,borderline south={2pt}{-2pt}{teal!50!black},}{axi}
\newcommand{\axref}[1]{{\bfseries\sffamily Axiom~\ref{axi:#1}}}

\newtcbtheorem[use counter from=thm]{lem}{Lemma}%補題
{mytheo,
	colframe=green!50!black,colback=green!50!black!2!white,colbacktitle=green!50!black,borderline south={2pt}{-2pt}{green!50!black},}{le}
\newcommand{\lemref}[1]{{\bfseries\sffamily Lemma~\ref{le:#1}}}

\newtcbtheorem[use counter from=thm]{con}{Conjection}%予想
{mytheo,
	colframe=magenta,colback=magenta!2!white,colbacktitle=magenta,borderline south={2pt}{-2pt}{charcoal},}{cn}
\newcommand{\conref}[1]{{\bfseries\sffamily Conjection~\ref{cn:#1}}}

%----------------------------------------------------------------------

%\newcounter{myexample}
%[usecounter=myexample,numberformat=\Alph]
%[number within=section]
\definecolor{charcoal}{rgb}{0.21, 0.27, 0.31}
\newtcbtheorem[use counter from=thm]{ex}{Example}%例
{mytheo,
	colframe=charcoal,colback=charcoal!2!white,colbacktitle=charcoal,borderline south={2pt}{-2pt}{charcoal},}{exa}
\newcommand{\exref}[1]{{\bfseries\sffamily Example~\ref{exa:#1}}}

\newtcbtheorem[use counter from=thm]{exe}{Exercise}%例
{mytheo,
	colframe=charcoal,colback=charcoal!2!white,colbacktitle=charcoal,borderline south={2pt}{-2pt}{charcoal},}{exer}
\newcommand{\exeref}[1]{{\bfseries\sffamily Exercise~\ref{exer:#1}}}

\newtcbtheorem[use counter from=thm]{qu}{Question}%例
{mytheo,
	colframe=charcoal,colback=charcoal!2!white,colbacktitle=charcoal,borderline south={2pt}{-2pt}{charcoal},}{que}
\newcommand{\quref}[1]{{\bfseries\sffamily Question~\ref{que:#1}}}

%----------------------------------------------------------------------

\newtcbtheorem[use counter from=thm]{thy}{Theory}%理論
{mytheo,
	colframe=yellow!50!black,colback=yellow!50!black!2!white,colbacktitle=yellow!50!black,borderline south={2pt}{-2pt}{yellow!50!black},}{ty}
\newcommand{\thyref}[1]{{\bfseries\sffamily Theory~\ref{ty:#1}}}

\newtcbtheorem[use counter from=thm]{pr}{Principle}%性質,原理
{mytheo,
	colframe=yellow!50!black,colback=yellow!50!black!2!white,colbacktitle=yellow!50!black,borderline south={2pt}{-2pt}{yellow!50!black},}{pri}
\newcommand{\prref}[1]{{\bfseries\sffamily Principle~\ref{pri:#1}}}

\newtcbtheorem[use counter from=thm]{req}{Requestment}%要請
{mytheo,
	colframe=yellow!50!black,colback=yellow!50!black!2!white,colbacktitle=yellow!50!black,borderline south={2pt}{-2pt}{yellow!50!black},}{rq}
\newcommand{\reqref}[1]{{\bfseries\sffamily Requestment~\ref{rq:#1}}}

\newtcbtheorem[use counter from=thm]{law}{Law}%法則
{mytheo,
	colframe=yellow!50!black,colback=yellow!50!black!2!white,colbacktitle=yellow!50!black,borderline south={2pt}{-2pt}{yellow!50!black},}{la}
\newcommand{\lawref}[1]{{\bfseries\sffamily Law~\ref{la:#1}}}

%----------------------------------------------------------------------

\newtcbtheorem[use counter from=thm]{expt}{Experiment}%実験
{mytheo,
	colframe=violet!50!black,colback=violet!50!black!2!white,colbacktitle=violet!50!black,borderline south={2pt}{-2pt}{violet!50!black},}{xp}
\newcommand{\exptref}[1]{{\bfseries\sffamily Experiment~\ref{xp:#1}}}

\newtcbtheorem[use counter from=thm]{res}{Resalt}%結果
{mytheo,
	colframe=purple!50!black,colback=purple!50!black!2!white,colbacktitle=purple!50!black,borderline south={2pt}{-2pt}{purple!50!black},}{rs}
\newcommand{\resref}[1]{{\bfseries\sffamily Resalt~\ref{rs:#1}}}

%\newtcbtheorem[use counter from=thm]{}{}%
%{mytheo}{}
%\newcommand{\_ref}[1]{{\bfseries\sffamily ~\ref{:#1}}}


%? ボックステンプレ tatebou
\tcbset{
	tatebou/.style={
		detach title,
		fonttitle=\gtfamily\sffamily\bfseries\upshape,
		coltitle=black,
		before skip=6pt,after skip=6pt, %ボックス上下の余白
		separator sign={\ }, %命題番号の後ろの記号
		description delimiters={(}{)\ }, %タイトルをどういう記号で囲むか
		after title={\ :\ \ }, %タイトルの後ろの記号
		before upper={\tcbtitle \setlength{\parindent}{1\zw}},%本文開始前につける記号など
		% after upper=\hfill\qed %ボックス末尾につける記号など
	}
}

\newtcolorbox{proof}{
%	invisible,
%	ignore,
%	breakable,
	tatebou,
	blanker,
	empty,
	title=Proof,
	left=3mm,
	right=3mm,
	top=1mm,
	bottom=1mm,
	borderline vertical={0.8pt}{0pt}{
		black,
		dotted,
%		arrows={Square[scale=0.5]-Square[scale=0.5]}
		},
	after upper=\hfill\square%ボックス末尾につける記号
}

\newtcolorbox{pf}{
%	invisible,
%	ignore,
%	breakable,
	tatebou,
	blanker,
	empty,
	title=Pf,
	left=3mm,
	right=3mm,
	top=1mm,
	bottom=1mm,
	borderline vertical={0.8pt}{0pt}{
		black,
		dotted,
%		arrows={Square[scale=0.5]-Square[scale=0.5]}
		},
	after upper=\hfill\square%ボックス末尾につける記号
}

\newtcolorbox{answer}{
%	invisible,
%	ignore,
%	breakable,
	tatebou,
	blanker,
	empty,
	title=Answer,
	left=3mm,
	right=3mm,
	top=1mm,
	bottom=1mm,
	borderline vertical={0.8pt}{0pt}{
		black,
		dotted,
%		arrows={Square[scale=0.5]-Square[scale=0.5]}
		},
	after upper=\hfill\square%ボックス末尾につける記号
}

\newtcolorbox{remark}[1][]{
	% invisible,
	% ignore,
	enhanced,
	before skip=2mm,after skip=3mm,fontupper=\gtfamily\sffamily,
	boxrule=0.4pt,left=5mm,right=2mm,top=1mm,bottom=1mm,
	colback=yellow!50,
	colframe=yellow!20!black,
	sharp corners,rounded corners=southeast,arc is angular,arc=3mm,
	underlay={%
		\path[fill=tcbcolback!80!black] ([yshift=3mm]interior.south east)--++(-0.4,-0.1)--++(0.1,-0.2);
		\path[draw=tcbcolframe,shorten <=-0.05mm,shorten >=-0.05mm] ([yshift=3mm]interior.south east)--++(-0.4,-0.1)--++(0.1,-0.2);
		\path[fill=yellow!50!black,draw=none] (interior.south west) rectangle node[white]{\Huge\bfseries !} ([xshift=4mm]interior.north west);
	},
	drop fuzzy shadow,#1}

\newcommand{\mrk}[1]{
	\!\!\!\!\!\!
	\raisebox{-3pt}{
		\begin{tikzpicture}[x=1pt, y=1pt, line width=1pt]
			\draw (0, 0) ellipse [x radius=4.5, y radius=6];
			\node at (0, 0){
				\usefont{T1}{m}{n}
				\fontsize{9pt}{0}
				\selectfont #1 };
\end{tikzpicture}}}