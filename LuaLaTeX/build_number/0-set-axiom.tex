\chapter{基礎知識}

ここでは基礎的な集合論や公理系などの準備をします.

具体的には

もう分かってる人や,逆に何言っているのかよくわからない人は飛ばしても大丈夫です.
ただし,ざっとは見てください.

\section{心構え}

\section{論理}

\newpage
\section{集合論}

集合論としては\(\mathtt{ZFC}\)集合論を仮定します.
要は大体普通の集合論ということです.

まずは \(\mathtt{ZF}\)集合論を作ります.わからなくていいです.

\begin{ax}{ZF}{ZF}
	以下の公理系を \emph{\(\mathtt{ZF}\)集合論 (Zermelo--Fraenkel set theory)} という.
	ただし,\(\phi,\varphi\)は自由変数に\(B\)を持たない任意の論理式である.
	\begin{align}
		&\exists x \forall y \lnot (y \in x).\\
		&\forall x \forall y ( \forall z ( z \in x \leftrightarrow z \in y ) \to x = y).\\
		&\forall x [ \exists a ( a \in x)
			\to \exists y (y \in x \land \lnot \exists z( z \in x \land z \in y))].\\
		&\forall A \forall w_1,\ldots,w_n\exists B \forall x (
			x\in B \leftrightarrow [y \in A \land \phi]).\\
		&\forall \mathcal{F} \exists A \forall Y \forall x
			[(x \in Y \land Y \in \mathcal{F}) \to x \in A ].\\
		&\forall A \forall w_1,\ldots,w_n
			[\forall x ( x \in A \to \exists ! y\varphi )
			\to \exists B \forall x (x \in A \to \exists y (y \in B \land \varphi))].\\
		&\exists X[\exists e(\forall z \lnot(z \in e)\land e \in X)
			\land \forall x (x \in X \to \exists y(y \in X \land \forall a (a \in y \to (a \in x \lor a = x))))].\\
		&\forall A \exists P \forall B [B \in P \leftrightarrow \forall z (z \in B \to z \in A)].
	\end{align}
\end{ax}

ここからまずは基本的な集合論の記号を整理していきます.それによって少し\(\mathtt{ZF}\)を書き直していきます.

\begin{dfn}{集合と元}{set}
	\axref{ZF}を満たす``ものの集まり''のことを\emph{集合 (set)}という.
	集合\(X\)に含まれる個々の``もの\(x\)'',
	つまり\(x \in X\)と書かれる\(x\)のことを\(X\)の\emph{元 (element)}という.
\end{dfn}

\begin{dfn}{部分集合}{subset}
	集合\(A,B\)について\(x \in A \to x \in B\) が成り立つとき,
	\(A\)は\(B\)の\emph{部分集合 (subset)}といい,\(A \subset B\)と書く.
\end{dfn}

\begin{dfn}{内包的記法}{intension-notation}
	\axref{ZF} (4) より,論理式\(\phi(x)\)を満たす要素だけを集めた集合を\(\set{x | \phi(x)}\)と表記する.
\end{dfn}

特に重要となる公理を挙げておきます.

\begin{dfn}{空集合}{emptyset}
	\axref{ZF} (1) より,いかなる要素も持たない集合が存在する.その集合を\(\varnothing\)と書く.
\end{dfn}

\begin{ax}{無限公理}{axiom-of-infinity}
	空集合\(\varnothing\)を含み,なおかつ\(n\)を含むときに\(n \cup \{n\}\)も含むような集合が存在する.
\end{ax}

\begin{dfn}{写像}{mapping}
	content...
\end{dfn}

\section{関係}

\begin{dfn}{同値関係・同値類}{equivalence-relation}
	content...
\end{dfn}

\begin{dfn}{順序}{order}
	content...
\end{dfn}
\begin{ex}
	content...
\end{ex}