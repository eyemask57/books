\section*{はじめに}

皆さん,「数」(数学ではよく「すう」と読みます)
って何だろうって考えたことありませんか? というかよく周りで聞きますよね,
虚数は現実世界に存在しないとかで殴り合っているとか,
よくわかんないけど四元数っていうすごい数があるだとか,
無限や無限小があるとかないとか,なんとか.

僕も数とはよくわかりません.
特に,何をもってこの世界に数が存在していると言えるのか分かりません.

全体として少なくとも一つ言えるのは,
数学や物理で用いられている数は\emph{作ることができる}ということです.
その中で成り立つ足し算も,掛け算も完備.今まで使っていた数と何ら変わらないものが作れちゃう.
もしかしたら本当の意味で本物でないにしても,少なくとも構造を模倣した\emph{数のおもちゃ}ぐらいは作れてしまいます.
\\\par
皆さん知ってました? 数って作れるんですよ.なら一回作って見たくないですか?

ということで,数の構成をまとめて置いといたらきっといい気がしたので書きます.

\section*{本書の方針}
本書の方針としては「基礎から」を想定しています.
つまり最も前提である公理系から定め,定理を証明していくことによって定義の正当性を担保していきます.
これにより,精密で緻密な数学の世界を眺めます.
かと言って数学に寄り添いすぎて分かりやすさを捨てるつもりはなく,
イメージや図などを多用して理解のしやすさを主軸をおきます.まとめると,
\begin{itemize}
	\item 大前提の公理系から出発する.
	\item 数学的な曖昧さをなるべく残さない.
	\item 分かりやすさを諦めない.
\end{itemize}
ということになります.分かりやすさを諦めたらごめんなさい.

本書では哲学的な「何故?」には踏み込みません.
そもそも数学とは,厳密でピースの数が膨大,何が問題か命題か自分で見つけ,
場合によっては正しいピースを作り,各々の答えを正しく求めるパズルです.
そして,その見つけたものの意味を明示的に論じるものではないのです.
まぁ,要は便利に使えればいいのです.
「数の構成」は数学の骨組みとなる厳密性を担保するいちピースとして使えます.
何より,「数って作れるんだ!」という感動をくれます.
それだけでもいい,と僕は思います.